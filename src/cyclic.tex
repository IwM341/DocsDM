Если ТМ взаимодействует с собой слабо, то можно не учитывать ее столкновения, и уравнение Больцмана станет линейным. Поскольку планету мы считаем изотропной, то ни интеграл столкновений, ни левая часть не будут зависеть от направления радиус вектора. Тогда по телесному углу в пространстве можно усреднить. Таким же образом можно усреднить по углу $\phi$ скорости. Останется только переменные $r$,$v_r$,$L$.
Уравнение Больцмана примет вид:

\begin{equation}
\label{eq:boltsman}
\frac{df}{dt} = C\left( {r,v_{r},L} \right) + St f( r,v_{r}',L') \left( r,v_{r},L \right)
\end{equation}

Еще одной циклической переменной, от которой нужно избавится, является параметр обиты (любая орбита определяется переменными $E$,$L$). Это может быть угол в полярных координатах либо время траектории $\tau$. Тогда $f$ выражается следующим образом

\begin{equation}
\label{eq:coord_new}
f(E,L,\tau)  = f(r(E,L,\tau),v_{r}(E,L,\tau),L)
\end{equation}

Введем также операцию усреднения по периоду $T$ (или большому промежутку времени) и проведем циклическое интегрирование по времени $\tau$ (т.е. по траектории)

\begin{equation}
\label{eq:cyclic_integral}
{\oint{\frac{df}{dt}d\tau}} = \frac{1}{T}{\oint{\frac{df}{dt}dt}} = \frac{f\left( {t + T,r,v_{r},L} \right) - f\left( t,r,v_{r},L \right)}{T} = \left\langle \frac{\partial f}{\partial t} \right\rangle_{T}
\end{equation}

Теперь ни правая ни левая части не зависят от $\tau$ и уравнение принимает вид
\begin{equation}
\label{eq:boltsman_EL}
\frac{\partial f}{\partial t} = C\left( {E,L} \right) + St\lbrack f\rbrack(E,L)
\end{equation}
	 
