Количество атомов водорода (неионизированнх) и свободных электронов можно очценить с помощбью формула Саха
\begin{equation*}
	\label{eq:sakha}
	\frac{n_e n_p}{n_A} = 
		\frac{g_e g_p}{g_A} (2\pi m_e T)^{3/2}e^{-J/T}
\end{equation*}

Безразмерный параметр $f = n/m^3$ - концентрация, деленая на куб массы.
Через этот параметр выразим степень ионизации $c_e$.

\begin{equation*}
	\label{eq:ion_degree_eq}
	\frac{c_e^2}{1-c_e} = \frac{1}{s} = \frac{g_e g_p}{g_A} \frac{1}{f_p}
	\left(\frac{2\pi T}{m}\right)^{3/2}e^{-J/T}
\end{equation*}

Отсюда

\begin{align*}
	\label{eq:ion_degree}
	c_e = \frac{2}{1+\sqrt{1+4s}}\\
	c_A = 1-c_e = \frac{4s}{(1+\sqrt{1+4s})^2}
\end{align*}

