Полная скорость захвата это интеграл
\begin{equation*}
	\label{eq:capture_rate}
	C_{+} = {\int{d^{3}\vec{r} \cdot d^{3}\vec{v}f_{k}\left( {r,v} \right) \cdot n_{p}f_{B}\left( {\vec{v}}_{1} \right)d^{3}{\vec{v}}_{1} \cdot \Gamma_{с}\left( \vec{v},{\vec{v}}_{1},r \right)}}
\end{equation*}
где

\begin{equation*}
	\label{eq:Gamma_def}
	\Gamma_{с}\left( {\vec{v},{\vec{v}}_{1},r} \right) = {\int\limits_{v' < v_{esc}}{d^{3}{\vec{v}}\,'\delta\left( {E_{f} - E_{in}} \right) \cdot \frac{m_{k}^{3}\left| \mathcal{M} \right|^{2}}{64\pi^{2}m_{i}^{2}m_{k}^{2}}}}
\end{equation*}
а матричный элемент равен

\begin{equation*}
	\left| \mathcal{M} \right|^{2} = 16G_{F}^{2}m_{i}^{2}m_{k}^{2} \cdot \Phi\left( q^{2} \right)dF
\end{equation*}

\begin{equation*}
	dF = \begin{cases*}
		1 &elastic \\
		\cfrac{s^{2}}{3}I^{2}(n)dn & migdal\\
		\cfrac{s^{2}}{3}I^{2}(\phi)\cfrac{d\phi}{2\pi} & ionization
	\end{cases*}
\end{equation*}

\begin{equation*}
	\Phi = \begin{cases*}
		1 & scalar-scalar\\
		\cfrac{- {q^{2}/2}}{m_{p}^{2}~or~m_{k}^{2}} & scalar-pseudoscalar\\
		\cfrac{q^{4}/4}{m_{p}^{2}m_{k}^{2}} & pseudoscalar-pseudoscalar
	\end{cases*}
\end{equation*}

Сечение ищется в системе ц.м. 
\begin{align*}
	\vec{V} = \frac{m_{i}{\vec{v}}_{1} + m_{k}\vec{v}}{m_{i} + m_{k}}\\
	\vec{\nu} = \frac{m_{i}}{m_{i} + m_{k}}\left( {\vec{v} - {\vec{v}}_{1}} \right)
\end{align*}

\begin{equation*}
	\vec{\nu} - {\vec{\nu}}\,' = \frac{m_{i}}{m_{i} + m_{k}}\left( {\vec{v} - {\vec{v}}\,' + \vec{v}_1^{\,\prime}- {\vec{v}}_{1}} \right) = \frac{m_{i}}{m_{i} + m_{k}}\left( {\vec{v} - {\vec{v}}\,'} \right)\left( {1 + \frac{m_{k}}{m_{i}}} \right) = \left( {\vec{v} - {\vec{v}}\,'} \right)
\end{equation*}

В такой замене переданный импульс равен
\begin{equation*}
	\vec{q} = m_{k}\left( \vec{\nu} - {\vec{\nu}}\,' \right)
\end{equation*}

Сечение соударений тогда равно
\begin{equation*}
	\Gamma_{с}\left( {\vec{v},{\vec{v}}_{1},r} \right) = \frac{m_{i}}{m_{k}\left( m_{i} + m_{k} \right)}4\pi\nu'd\vec{n}\,' \cdot \frac{m_{k}^{3}\left| \mathcal{M} \right|^{2}}{64\pi^{2}m_{i}^{2}m_{k}^{2}} = \nu' d\vec{n}\,'\frac{G_{F}^{2}}{\pi}\frac{m_i m_{k}^{2}}{\left( m_{i} + m_{k} \right)}\Phi dF
\end{equation*}

А общая скорость захвата имеет вид
\begin{equation}
	\label{eq:capture_rate_result}
	C_+ = Vn_{\chi}\frac{G_{F}^{2}}{\pi}\frac{m_{i} m_{k}^{2}}{\left( {m_{i} + m_{k}} \right)}~d\xi \cdot f_{rm}\left( \alpha_{v} \right)d\alpha_{v} \cdot n_{i}f_{B}^{rm}d\omega dc_{1} \cdot \nu' d\vec{n}\,'\Phi\left( q^{2} \right)dF
\end{equation}

Первая часть --- размерные множетели, вторая --- безразмерные части интгрирования методом монте-карло. $d\xi$ --- выбор координаты $r$, $f_{rm}\left( \alpha_{v} \right)d\alpha_{v}$ --- выбор скорости частицы т.м., $n_{i}$ --- концентрация мишени в точке, $f_{B}^{rm}d\omega dc_{1}$ --- выбор скорости мишени, $\nu' d\vec{n}\,'\Phi\left( q^{2} \right)dF$ --- выбор выходной скорости.

Концентрация элемента $i$ равна
\begin{equation*}
	n_{i}(r) = \cfrac{\rho(r) \widetilde{\rho}_i(r)}{m_i} = 
	\overline{\rho}\cfrac{\widehat{\rho}(r) \widetilde{\rho}_i(r)}{m_i}
\end{equation*}
Итоговым результатом будет безразмерная скорость захвата $c_+$, тогда
\begin{align*}
	C_+ = \sum{c_{i+}} \cdot \left[ V\overline{\rho}n_{\chi}\frac{G_{F}^{2}}{\pi}m_k
	\right]
	\\
	c_{i+} = \frac{m_k}{m_k+m_i}d\xi \cdot f_{rm}\left( \alpha_{v} \right)d\alpha_{v} \cdot \widehat{\rho}(r) \widetilde{\rho}_i(r) f_{B}^{rm}d\omega dc_{1} \cdot \nu' d\vec{n}\,'\Phi\left( q^{2} \right)dF
\end{align*}

Для поиска распределения по энергии и импульсу нужно при интегрировании  найти выходную скорость и добавить соответствующий вес в гистограмму.




