Мы не будем учитывать неоднородности сферического тела по угловым координатам, поэтому уравнения движения и фазовая плотность зависит только от трех переменных: скорость $v$, радиус $r$ и орбитальный момент $L=rv$.
Фазовый объем в новых переменных выглядит следующим образом:



\begin{equation}
\label{eq:phase_volume_v}
d\Phi = d^{3}\vec{x}d^{3}\vec{v} = 4\pi r^{2}dr \cdot \frac{2\pi vdvdL^{2}}{r\sqrt{r^{2}v^{2} - L^{2}}}
\end{equation}
Можно взять вместо скорости $v$ радиальную скорость $v_r$ и тогда фазовый объем станет следующим:
\begin{equation}
\label{eq:phase_volume_v_r}
	d\Phi = 4\pi r^{2}dr \cdot \frac{2\pi dv_{r}dL^{2}}{r^{2}} = 8\pi^{2}drdv_{r}dL^{2}
\end{equation}
Однородный потенциал $\phi(r)$ возьмем положительным. Тогда уравнение движения и закон сохранения энергии будут следующими:
\begin{equation}
\label{eq:phase_volume_v_r}
	\dot{\vec{r}}= - \nabla\phi(r) \\
	E = \frac{v^{2}}{2} + \phi(r)
\end{equation}
