Интеграл столкновений превратится в матрицу столкновений, элементами которой будут вероятности за единицу времени перейти из одного бина в другой. Помимо матрицы столкновений есть еще вектор испарения (туда попадает то, что не попало в матрицу столкновения).

Эта величина считается точно так же как и скорость захвата (\ref{eq:capture_rate_result}), но та часть интеграла, которая выбирает скорость и положение частицы $Vn_{\chi}d\xi f_{rm}\left( \alpha_{v} \right)d\alpha_{v}$, изменяется на выражение, определяющее положение и направление движения частицы при известном $e$ и $l$
\begin{equation}
	\cfrac{T_{in}}{T_{in}+T_{out}}\cdot d\tau d\vec{n}
\end{equation}
где $T_{in}$ --- часть периода внутри тела, $T_{out}$ --- снаружи, $d\tau$ --- выбор времени 

В итоге получается
\begin{equation}
	\label{eq:st}
	ST = \cfrac{T_{in}}{T_{in}+T_{out}}\cdot d\tau d\vec{n} \cdot \overline{n}_{p}\frac{g_{F}^{2}}{\pi}\frac{m_{k}^{2}}{m_{p}^{3}\left( {m_{p} + m_{k}} \right)}~ \cdot {\widetilde{n}}_{p}f_{B}^{rm}d\omega dc_{1} \cdot \nu' d\vec{n}\,'\Phi\left( q^{2} \right)dF
\end{equation}

Интеграл столкновений тоже обезразмерим, выкинув массы и константы
\begin{equation*}
	ST = st \cdot\overline{n}_{p} \frac{g_{F}^{2}}{\pi}\frac{m_{k}^{2}}{m_{p}^{3}\left( {m_{p} + m_{k}} \right)}
\end{equation*}

